\documentclass[\main.tex]{subfiles}

\chapter{Conclusão}

\section{Reflexão Crítica dos Resultados}
A solução desenvolvida é adequada às necessidades requeridas pelo projeto. Em relação aos
resultados, existem sempre bastantes problemas com alguns modelos, visto que falham imenso
por falta de convergência de alguns pontos. Na duração do projeto foi utilizada uma
biblioteca, que no momento se encontra descontinuada 
(\textit{\say{statsmodels.tsa.arima\_model.ARIMA}}), todavia já não era possível fazer a
migração para a nova versão pela incompatibilidade com o \textit{\gls{google colab}}.
Foi exprimentada a nova biblioteca localmente e alguns dos modelos que anteriormente não
obtinham resultados por falta de convergência de pontos, nesta biblioteca já conseguiam
ter sucesso nas previsões.\par
Abordando os modelos de séries temporais estudadas, existem bastante utilidade nas
possíveis previsões a realizar com os mesmos, visto que, é possível prever dados
importantíssimos que normalmente utilizamos (ou seria vantajoso se o fizéssemos) como a
velocidade média numa certa estrada ou o tráfego de uma estrada a certas horas do dia.\par
Debatendo sobre o segundo \textit{\gls{dataset}} estudado, as previsões a realizar nesta
altura que estamos a viver com a \gls{covid}, seria bastante interessante ter estimativas
da ocupação de certos espaços com base em dados anteriores e poderia oferecer uma ajuda
enorme no assunto.

\section{Conclusão e Trabalho Futuro}
No futuro, uma implementação essêncial seria a utilização da biblioteca na sua versão mais
recente, para que mais modelos possam terminar com sucesso e, desse modo, conseguir
melhores previsões e mais precisas.\par
Em relação à incompatibilidade do \textit{\gls{google colab}} com esta biblioteca poderia
ser contorado com o desenvolvimento de uma \textit{\acrshort{api}} e implementação do
ambiente virtual de \textit{\gls{python}} num servidor, com a versão da linguagem
utilizada neste projeto e as dependências necessárias para a nova versão da biblioteca.
Ao servir essa \textit{\acrshort{api}} com esse servidor, teriamos uma plataforma de
testes completamente funcional e com as bibliotecas necessárias mais atuais para um maior
sucesso na execução dos algoritmos.\par
Em suma, o trabalho foi concluído com sucesso e, na opinião do autor, as ferramentas
trabalhadas podem vir a ser bastante úteis futuramente. O autor também espera concluir
o trabalho futuro discutido para que o projeto fique cada vez mais único e útil.

\newpage