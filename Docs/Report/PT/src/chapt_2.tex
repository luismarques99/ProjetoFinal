\documentclass[\main.tex]{subfiles}

\chapter{Concetualização do Problema}
\begin{singlespace}
\minitoc
\end{singlespace}
\vspace{20pt}

\section{Requisitos}
Ao longo da realização do projeto, foram sendo definidos os requisitos para o
desenvolvimento do mesmo, sendo eles:
\begin{itemize}
    \item Estudar modelo \textit{\acrshort{arima}} e variantes;
    \item Construir um serviço de testes de \textit{\glspl{dataset}} com os modelos
    estudados anteriormente;
    \item Testar varios \textit{\glspl{dataset}} e guardar melhores modelos e configurações
    dos mesmos;
    \item Integrar projeto com \textit{\gls{google colab}}.
\end{itemize} \par

\indent Destas necessidades definidas em cima, podemos retirar algumas tarefas e subtarefas
implícitas em cada uma.\par

\subsection{Estudo do Modelo}
\begin{itemize}
    \item Estudo de algumas bibliotecas de \textit{\gls{python}} utilzadas ao longo do
    trabalho;
    \item Realização de alguns exercícios de \textit{\acrlong{ml}} com
    \textit{\gls{python}} (e.g., abrir e manusear um \textit{\gls{dataset}}, formatar as
    datas, imprimir ou exportar os dados, mostrar ou exportar um gráfico);
    \item Estudo do modelo \textit{\acrshort{arima}} (e.g., propriedades, funcionamento,
    características)
    \item Estudo das variantes do modelo \textit{\acrshort{arima}} que são:
    \textit{\acrshort{arimax}} (Média Móvel Integrada Autoregressiva com \gls{variaveis
    exogenas}), \textit{\acrshort{sarima}} (Média Móvel Integrada Autoregressiva Sazonal)
    e \textit{\acrshort{sarimax}} (Média Móvel Integrada Autoregressiva Sazonal com
    \gls{variaveis exogenas}).
\end{itemize}\par

\subsection{Construção do Serviço de Testes}
\begin{itemize}
    \item Estruturação dos \textit{\glspl{script}};
    \item Realização dos \textit{\glspl{script}} na linguagem \textit{\gls{python}};
    \item Testar a veracidade e qualidade do programa desenvolvido.
\end{itemize}\par

\subsection{Teste dos Datasets}
\begin{itemize}
    \item Executar uma \textit{\gls{grid search}} para encontrar melhores modelos e melhores
    configurações;
    \item Executar os melhores modelos com as melhores configurações e guardar resultados.
\end{itemize}\par

\subsection{Integração com \textit{Google Colab}}
\begin{itemize}
    \item Estudo da plataforma \textit{\gls{google colab}} (e.g., funcionalidades,
    compatiblidade);
    \item Colocar \textit{\glspl{script}} na plataforma;
    \item Correr alguns testes.
\end{itemize}\par
\vspace{5pt}


\section{Arquitetura Concetual}
«Arquitetura concetual - Nesta arquitetura deverá ser claro o que foi efetivamente
desenvolvido pelo Estudante e aquilo que foi desenvolvido por terceiros. A arquitetura
deverá realçar aspetos relacionadas com integração, protocolos, entre outros, e que o
estudante deverá clarificar.»

\newpage