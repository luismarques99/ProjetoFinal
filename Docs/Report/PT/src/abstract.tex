\documentclass[\main.tex]{subfiles}

\chapter*{Resumo}

De modo a realizar uma análise comparativa entre diferentes algoritmos de aprendizagem com
base em séries temporais, é necessário abordar estes dois conceitos desagregadamente e
estudar cada um deles individualmente.\par
Este projeto representa um estudo realizado em torno de alguns modelos, entre eles o modelo
\textit{\acrshort{arima}}, \textit{\acrshort{arimax}}, \textit{\acrshort{sarima}} e
\textit{\acrshort{sarimax}}. A estrutura de cada algoritmo foi desenhada utilizando alguns
recursos disponibilizados pelo Professor Fábio Silva e com a ajuda de algumas bibliotecas
da linguagem \textit{\gls{python}}, sendo as mais evidentes a biblioteca
\textit{\say{pandas}}, \textit{\say{numpy}}, \textit{\say{sklearn}}, 
\textit{\say{statsmodels}} e \textit{\say{matplotlib}}.\\

\vspace{5pt}

\noindent\textbf{Palavras Chave}: \textit{\acrshort{arima}}, \acrlong{ia},
\textit{\acrlong{ml}}, \textit{\gls{python}}

\newpage